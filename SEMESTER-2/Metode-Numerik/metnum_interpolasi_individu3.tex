\documentclass{article}
\usepackage{graphicx} % Required for inserting images
\usepackage{amsmath}
\usepackage[a4paper, margin=0.5in]{geometry}
\renewcommand\thesubsection{\alph{subsection}}
\renewcommand*{\thesubsubsection}{-{}}

\title{Metode Numerik : Interpolasi Polinomial}
\author{Muhammad Akmal Fazli Riyadi / Kelas D}
\date{24060124130123}

\begin{document}

\maketitle

\section{Diberikan data sebagai berikut :}

\begin{center}
\begin{tabular}{|c|c|}
    \hline
    $x$ & $f(x)$ \\
    \hline
    1.0 & 1.63 \\
    1.2 & 2.20 \\
    1.4 & 2.81 \\
    1.6 & 3.62 \\
    1.8 & 4.43 \\
    \hline
\end{tabular}
\end{center}

\subsection{Tentukan nilai interpolasi polinomial metode Newton Beda Maju untuk $P_n(1.1)$ dan
tentukan galat absolutnya, jika diasumsikan y = f(x) = x.exp(0.5x)! Tunjukkan langkah-langkahnya!}

Selisih maju pertama:

\begin{align*}
    \Delta f(x) &= f(x_{i+1}) - f(x_i)
\end{align*}

\begin{center}
\begin{tabular}{|c|c|}
    \hline
    $x$ & $\Delta f(x)$ \\
    \hline
    1.0 & 0.57 \\
    1.2 & 0.61 \\
    1.4 & 0.81 \\
    1.6 & 0.81 \\
    \hline
\end{tabular}
\end{center}

Selisih maju kedua:

\begin{align*}
    \Delta^2 f(x) &= \Delta f(x_{i+1}) - \Delta f(x_i)
\end{align*}

\begin{center}
\begin{tabular}{|c|c|}
    \hline
    $x$ & $\Delta^2 f(x)$ \\
    \hline
    1.0 & 0.04 \\
    1.2 & 0.20 \\
    1.4 & 0.00 \\
    \hline
\end{tabular}
\end{center}

Selisih maju ketiga:

\begin{align*}
    \Delta^3 f(x) &= \Delta^2 f(x_{i+1}) - \Delta^2 f(x_i)
\end{align*}

\begin{center}
\begin{tabular}{|c|c|}
    \hline
    $x$ & $\Delta^3 f(x)$ \\
    \hline
    1.0 & 0.16 \\
    1.2 & -0.20 \\
    \hline
\end{tabular}
\end{center}

Selisih maju keempat:

\begin{align*}
    \Delta^4 f(x) &= \Delta^3 f(x_{i+1}) - \Delta^3 f(x_i)
\end{align*}

\begin{center}
\begin{tabular}{|c|c|}
    \hline
    $x$ & $\Delta^4 f(x)$ \\
    \hline
    1.0 & -0.36 \\
    \hline
\end{tabular}
\end{center}

\subsubsection{Polinomial Newton Beda Maju}
Rumus interpolasi Newton beda maju:

\begin{align*}
    P_n(x) &= f(x_0) + u\Delta f(x_0) + \frac{u(u-1)}{2!} \Delta^2 f(x_0) \\
    &+ \frac{u(u-1)(u-2)}{3!} \Delta^3 f(x_0) + \frac{u(u-1)(u-2)(u-3)}{4!} \Delta^4 f(x_0)
\end{align*}

Dengan:

\begin{align*}
    u &= \frac{x - x_0}{h}, \quad h = 0.2, \quad x_0 = 1.0
\end{align*}

Untuk $x = 1.1$:

\begin{align*}
    u &= \frac{1.1 - 1.0}{0.2} = 0.5
\end{align*}

Substitusi nilai:

\begin{align*}
    P_n(1.1) &= 1.63 + (0.5)(0.57) + \frac{(0.5)(-0.5)}{2!} (0.04) \\
    &+ \frac{(0.5)(-0.5)(-1.5)}{3!} (0.16) + \frac{(0.5)(-0.5)(-1.5)(-2.5)}{4!} (-0.36)
\end{align*}

Hitung tiap suku:

\begin{align*}
    P_n(1.1) &= 1.63 + 0.285 + (-0.005) + 0.02 + (-0.0075)
\end{align*}

\begin{align*}
    P_n(1.1) &\approx 1.9225
\end{align*}

\subsubsection{Galat Absolut}
Nilai sebenarnya:

\begin{align*}
    f(1.1) &= 1.1 \times e^{0.5 \times 1.1}
\end{align*}

\begin{align*}
    f(1.1) &= 1.1 \times e^{0.55} \approx 1.1 \times 1.733 = 1.906
\end{align*}

Galat absolut:

\begin{align*}
    | f(1.1) - P_n(1.1) | &= | 1.906 - 1.9225 |
\end{align*}

\begin{align*}
    &= 0.0165
\end{align*}

\subsubsection{Kesimpulan}
Nilai interpolasi Newton Beda Maju untuk $P_n(1.1)$ adalah \textbf{1.9225}, dengan galat absolut sekitar \textbf{0.0165}.

\subsection{Tentukan nilai interpolasi polinomial metode Newton Beda Mundur untuk $P_n(1.7)$ dan
tentukan galat absolutnya, jika diasumsikan y = f(x) = x.exp(0.5x)! Tunjukkan langkah-langkahnya!}

Selisih mundur pertama:

\begin{align*}
    \Delta f(x) &= f(x_i) - f(x_{i-1})
\end{align*}

\begin{center}
\begin{tabular}{|c|c|}
    \hline
    $x$ & $\Delta f(x)$ \\
    \hline
    1.2 & 0.57 \\
    1.4 & 0.61 \\
    1.6 & 0.81 \\
    1.8 & 0.81 \\
    \hline
\end{tabular}
\end{center}

Selisih mundur kedua:

\begin{align*}
    \Delta^2 f(x) &= \Delta f(x_i) - \Delta f(x_{i-1})
\end{align*}

\begin{center}
\begin{tabular}{|c|c|}
    \hline
    $x$ & $\Delta^2 f(x)$ \\
    \hline
    1.2 & 0.04 \\
    1.4 & 0.20 \\
    1.6 & 0.00 \\
    1.8 & 0.00 \\
    \hline
\end{tabular}
\end{center}

Selisih mundur ketiga:

\begin{align*}
    \Delta^3 f(x) &= \Delta^2 f(x_i) - \Delta^2 f(x_{i-1})
\end{align*}

\begin{center}
\begin{tabular}{|c|c|}
    \hline
    $x$ & $\Delta^3 f(x)$ \\
    \hline
    1.4 & 0.16 \\
    1.6 & -0.20 \\
    1.8 & -0.20 \\
    \hline
\end{tabular}
\end{center}

Selisih mundur keempat:

\begin{align*}
    \Delta^4 f(x) &= \Delta^3 f(x_i) - \Delta^3 f(x_{i-1})
\end{align*}

\begin{center}
\begin{tabular}{|c|c|}
    \hline
    $x$ & $\Delta^4 f(x)$ \\
    \hline
    1.6 & -0.36 \\
    1.8 & 0.00 \\
    \hline
\end{tabular}
\end{center}

\subsubsection{Polinomial Newton Beda Mundur}
Rumus interpolasi Newton beda mundur:

\begin{align*}
    P_n(x) &= f(x_n) + u\Delta f(x_n) + \frac{u(u+1)}{2!} \Delta^2 f(x_n) \\
    &+ \frac{u(u+1)(u+2)}{3!} \Delta^3 f(x_n) + \frac{u(u+1)(u+2)(u+3)}{4!} \Delta^4 f(x_n)
\end{align*}

Dengan:

\begin{align*}
    u &= \frac{x - x_n}{h}, \quad h = 0.2, \quad x_n = 1.8
\end{align*}

Untuk $x = 1.7$:

\begin{align*}
    u &= \frac{1.7 - 1.8}{0.2} = -0.5
\end{align*}

Substitusi nilai:

\begin{align*}
    P_n(1.7) &= 4.43 + (-0.5)(0.81) + \frac{(-0.5)(-1.5)}{2!} (0.00) \\
    &+ \frac{(-0.5)(-1.5)(-2.5)}{3!} (-0.20) + \frac{(-0.5)(-1.5)(-2.5)(-3.5)}{4!} (0.00)
\end{align*}

Hitung tiap suku:

\begin{align*}
    P_n(1.7) &= 4.43 - 0.405 + 0 + (-0.021) + 0
\end{align*}

\begin{align*}
    P_n(1.7) &\approx 4.004
\end{align*}

\subsubsection{Galat Absolut}
Nilai sebenarnya:

\begin{align*}
    f(1.7) &= 1.7 \times e^{0.5 \times 1.7}
\end{align*}

\begin{align*}
    f(1.7) &= 1.7 \times e^{0.85} \approx 1.7 \times 2.339 = 3.976
\end{align*}

Galat absolut:

\begin{align*}
    | f(1.7) - P_n(1.7) | &= | 3.976 - 4.004 |
\end{align*}

\begin{align*}
    &= 0.028
\end{align*}

\subsubsection{Kesimpulan}
Nilai interpolasi Newton Beda Mundur untuk $P_n(1.7)$ adalah \textbf{4.004}, dengan galat absolut sekitar \textbf{0.028}.

\section{Diketahui:}
\begin{center}
\begin{tabular}{|c|c|}
    \hline
    $x$ & $f(x)$ \\
    \hline
    0.4 & 2.50 \\
    0.6 & 3.51 \\
    0.7 & 4.12 \\
    1.0 & 6.03 \\
    \hline
\end{tabular}
\end{center}

\subsection{Tentukan nilai interpolasi polinomial metode Lagrange untuk $P_n(0.5)$! Tunjukkan
langkah-langkahnya!}

\subsubsection{Rumus Polinomial Lagrange}
Polinomial interpolasi Lagrange didefinisikan sebagai:
\[
P_n(x) = \sum_{i=0}^{n} L_i(x) f(x_i)
\]
dengan:
\[
L_i(x) = \prod_{\substack{j=0 \\ j\neq i}}^{n} \frac{x - x_j}{x_i - x_j}
\]

\subsubsection{Menghitung Lagrange Basis \( L_i(0.5)\)}
Basis \( L_0(0.5) \) 
\[
L_0(0.5) = \frac{(0.5 - 0.6)(0.5 - 0.7)(0.5 - 1.0)}{(0.4 - 0.6)(0.4 - 0.7)(0.4 - 1.0)}
\]

Basis \( L_1(0.5) \)
\[
L_1(0.5) = \frac{(0.5 - 0.4)(0.5 - 0.7)(0.5 - 1.0)}{(0.6 - 0.4)(0.6 - 0.7)(0.6 - 1.0)}
\]

Basis \( L_2(0.5) \)
\[
L_2(0.5) = \frac{(0.5 - 0.4)(0.5 - 0.6)(0.5 - 1.0)}{(0.7 - 0.4)(0.7 - 0.6)(0.7 - 1.0)}
\]

Basis \( L_3(0.5) \)
\[
L_3(0.5) = \frac{(0.5 - 0.4)(0.5 - 0.6)(0.5 - 0.7)}{(1.0 - 0.4)(1.0 - 0.6)(1.0 - 0.7)}
\]

\subsubsection{Substitusi ke dalam Rumus}
\[
P_n(0.5) = L_0(0.5) f(0.4) + L_1(0.5) f(0.6) + L_2(0.5) f(0.7) + L_3(0.5) f(1.0)
\]

Setelah perhitungan diperoleh:
\[
P_n(0.5) \approx 2.961
\]

Dengan demikian, nilai interpolasi polinomial metode Lagrange di \( x = 0.5 \) adalah \textbf{2.961}.

\subsection{Tentukan galat absolutnya, bila diasumsikan bahwa $f(x) = 2x^2 + 3x +1$. Tunjukkan langkah-langkahnya!}

Nilai interpolasi yang telah dihitung sebelumnya:
\[
P_n(0.5) \approx 2.961
\]

Substitusi  ke dalam fungsi asli:
\begin{align*}
f(0.5) &= 2(0.5)^2 + 3(0.5) + 1 
= 2(0.25) + 1.5 + 1 
= 0.5 + 1.5 + 1 
= 3.0
\end{align*}

Galat absolut didefinisikan sebagai:
\[
E = | f(0.5) - P_n(0.5) |
\]

Substitusi nilai yang telah dihitung:
\begin{align*}
E &= |3.0 - 2.961| = 0.0394
\end{align*}

Dengan demikian, galat absolut dari Interpolasi Lagrange adalah \textbf{0.0394}.

\section{Diketahui:}
\begin{center}
\begin{tabular}{|c|c|}
    \hline
    $x$ & $f(x)$ \\
    \hline
    0.4 & 2.53 \\
    0.6 & 3.50 \\
    0.7 & 4.11 \\
    1.0 & 6.02 \\
    1.2 & 7.53 \\
    \hline
\end{tabular}
\end{center}

\subsection{Tentukan nilai interpolasi polinomial metode Neville untuk $P_n(0.5)$! Tunjukkan langkah-langkahnya!}
Interpolasi dilakukan untuk $x = 0.5$ menggunakan metode Neville dengan formula:
\begin{align*}
    P_{i,j} = \frac{(x - x_{i-j}) P_{i,j-1} - (x - x_i) P_{i-1,j-1}}{x_i - x_{i-j}}
\end{align*}

\subsubsection{Perhitungan:}
Tingkat pertama ($j = 1$)
\begin{align*}
    P_{1,1} &= \frac{(0.5 - 0.4) \cdot 3.50 - (0.5 - 0.6) \cdot 2.53}{0.6 - 0.4} \\
            &= \frac{(0.1) \cdot 3.50 - (-0.1) \cdot 2.53}{0.2} \\
            &= \frac{0.35 + 0.253}{0.2} = 3.015
\end{align*}

\begin{align*}
    P_{2,1} &= \frac{(0.5 - 0.6) \cdot 4.11 - (0.5 - 0.7) \cdot 3.50}{0.7 - 0.6} \\
            &= \frac{(-0.1) \cdot 4.11 - (-0.2) \cdot 3.50}{0.1} \\
            &= \frac{-0.411 + 0.7}{0.1} = 2.89
\end{align*}

\begin{align*}
    P_{3,1} &= \frac{(0.5 - 0.7) \cdot 6.02 - (0.5 - 1.0) \cdot 4.11}{1.0 - 0.7} \\
            &= \frac{(-0.2) \cdot 6.02 - (-0.5) \cdot 4.11}{0.3} \\
            &= \frac{-1.204 + 2.055}{0.3} = 2.837
\end{align*}

Tingkat kedua ($j = 2$)
\begin{align*}
    P_{2,2} &= \frac{(0.5 - 0.4) \cdot 2.89 - (0.5 - 0.7) \cdot 3.015}{0.7 - 0.4} \\
            &= \frac{(0.1) \cdot 2.89 - (-0.2) \cdot 3.015}{0.3} \\
            &= \frac{0.289 + 0.603}{0.3} = 2.9733
\end{align*}

\begin{align*}
    P_{3,2} &= \frac{(0.5 - 0.6) \cdot 2.837 - (0.5 - 0.7) \cdot 2.89}{1.0 - 0.6} \\
            &= \frac{(-0.1) \cdot 2.837 - (-0.2) \cdot 2.89}{0.4} \\
            &= \frac{-0.2837 + 0.578}{0.4} = 2.8575
\end{align*}

Tingkat ketiga ($j = 3$)
\begin{align*}
    P_{3,3} &= \frac{(0.5 - 0.4) \cdot 2.8575 - (0.5 - 0.7) \cdot 2.9733}{1.0 - 0.4} \\
            &= \frac{(0.1) \cdot 2.8575 - (-0.2) \cdot 2.9733}{0.6} \\
            &= \frac{0.28575 + 0.59466}{0.6} \\
            &= 2.934
\end{align*}

\subsubsection{Kesimpulan}
Hasil interpolasi polinomial metode Neville untuk $P_3(0.5)$ adalah \textbf{2.934}.

\subsection{Tentukan galat absolutnya, bila diasumsikan bahwa $f(x) = 2x^2+3x+1$. Tunjukkan langkah-langkahnya!}
Diketahui fungsi asli:
\begin{align*}
    f(x) = 2x^2 + 3x + 1
\end{align*}

Hitung nilai eksak di $x = 0.5$:
\begin{align*}
    f(0.5) &= 2(0.5)^2 + 3(0.5) + 1 \\
           &= 2(0.25) + 1.5 + 1 \\
           &= 0.5 + 1.5 + 1 = 3
\end{align*}

Galat absolut dihitung sebagai:
\begin{align*}
    E &= |f(0.5) - P_3(0.5)| \\
      &= |3 - 2.934| \\
      &= 0.066
\end{align*}

Dengan demikian, galat absolut dari Interpolasi Neville adalah \textbf{0.066}.


\end{document}
